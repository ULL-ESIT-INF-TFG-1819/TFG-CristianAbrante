\documentclass[spanish,a4paper,12pt,oneside]{extreport}

%%%%%%%%%%%%%%%%%%%%%%%%%%%%%%%%%%%%%%%%%%%%%%%%%%%%%%%%%%%%%%%%%%%%%%%%%%%%%%%
\usepackage{graphicx}
\usepackage{subcaption}
\usepackage{float}
\usepackage[dvips]{epsfig}
\usepackage[utf8]{inputenc}
\usepackage[spanish]{babel}
\usepackage{alltt}
\usepackage{algorithm}
\usepackage{algorithmic}
\usepackage[]{algorithm2e}
\usepackage{multirow}
\usepackage{amsmath}
\usepackage{eurosym}
\usepackage[top=2cm, bottom=2cm, left=2cm, right=2cm]{geometry}
%%%%%%%%%%%%%%%%%%%%%%%%%%%%%%%%%%%%%%%%%%%%%%%%%%%%%%%%%%%%%%%%%%%%%%%%%%%%%%%

\usepackage{listings} %code highlighter
\usepackage{color} %use color
\definecolor{mygreen}{rgb}{0,0.6,0}
\definecolor{mygray}{rgb}{0.5,0.5,0.5}
\definecolor{mymauve}{rgb}{0.58,0,0.82}
 
%Customize a bit the look
\lstset{ %
backgroundcolor=\color{white}, % choose the background color; you must add \usepackage{color} or \usepackage{xcolor}
basicstyle=\footnotesize, % the size of the fonts that are used for the code
breakatwhitespace=false, % sets if automatic breaks should only happen at whitespace
breaklines=true, % sets automatic line breaking
captionpos=b, % sets the caption-position to bottom
commentstyle=\color{mygreen}, % comment style
deletekeywords={...}, % if you want to delete keywords from the given language
escapeinside={\%*}{*)}, % if you want to add LaTeX within your code
extendedchars=true, % lets you use non-ASCII characters; for 8-bits encodings only, does not work with UTF-8
frame=single, % adds a frame around the code
keepspaces=true, % keeps spaces in text, useful for keeping indentation of code (possibly needs columns=flexible)
keywordstyle=\color{blue}, % keyword style
% language=Octave, % the language of the code
morekeywords={*,...}, % if you want to add more keywords to the set
numbers=left, % where to put the line-numbers; possible values are (none, left, right)
numbersep=5pt, % how far the line-numbers are from the code
numberstyle=\tiny\color{mygray}, % the style that is used for the line-numbers
rulecolor=\color{black}, % if not set, the frame-color may be changed on line-breaks within not-black text (e.g. comments (green here))
showspaces=false, % show spaces everywhere adding particular underscores; it overrides 'showstringspaces'
showstringspaces=false, % underline spaces within strings only
showtabs=false, % show tabs within strings adding particular underscores
stepnumber=1, % the step between two line-numbers. If it's 1, each line will be numbered
stringstyle=\color{mymauve}, % string literal style
tabsize=2, % sets default tabsize to 2 spaces
title=\lstname % show the filename of files included with \lstinputlisting; also try caption instead of title
}
%END of listing package%
 
\definecolor{darkgray}{rgb}{.4,.4,.4}
\definecolor{purple}{rgb}{0.65, 0.12, 0.82}
 
%define Javascript language
\lstdefinelanguage{JavaScript}{
keywords={typeof, new, true, false, catch, function, return, null, catch, switch, var, if, in, while, do, else, case, break},
keywordstyle=\color{blue}\bfseries,
ndkeywords={class, export, boolean, throw, implements, import, this},
ndkeywordstyle=\color{darkgray}\bfseries,
identifierstyle=\color{black},
sensitive=false,
comment=[l]{//},
morecomment=[s]{/*}{*/},
commentstyle=\color{purple}\ttfamily,
stringstyle=\color{red}\ttfamily,
morestring=[b]',
morestring=[b]"
}
 
\lstset{
language=JavaScript,
extendedchars=true,
basicstyle=\footnotesize\ttfamily,
showstringspaces=false,
showspaces=false,
numbers=left,
numberstyle=\footnotesize,
numbersep=9pt,
tabsize=2,
breaklines=true,
showtabs=false,
captionpos=b
}

\newcommand{\SONY}{{\sc Sony}}
\newcommand{\MICROSOFT}{{\sc Microsoft}}
\newcommand{\GCC}{\textsf{\textsc{G}CC}}
\newcommand{\INTEL}{\textsf{\textsc{I}ntel}}

%%% Traducimos el pseudocodigo
\renewcommand{\algorithmicwhile}{\textbf{mientras}}
\renewcommand{\algorithmicend}{\textbf{fin}}
\renewcommand{\algorithmicdo}{\textbf{hacer}}
\renewcommand{\algorithmicif}{\textbf{si}}
\renewcommand{\algorithmicthen}{\textbf{entonces}}
\renewcommand{\algorithmicrepeat}{\textbf{repetir}}
\renewcommand{\algorithmicuntil}{\textbf{hasta que}}
\renewcommand{\algorithmicelse}{\textbf{en otro caso}}
\renewcommand{\algorithmicfor}{\textbf{para}}
\newcommand{\CC}{C\nolinebreak\hspace{-.05em}\raisebox{.4ex}{\tiny\bf +}\nolinebreak\hspace{-.10em}\raisebox{.4ex}{\tiny\bf +}}

%\newcommand{\RETURN}{\textbf{retornar} }
\newcommand{\RET}{\STATE \textbf{retornar} }
\newcommand{\TO}{\textbf{hasta} }
\newcommand{\AND}{\textbf{y} }
\newcommand{\OR}{\textbf{o} }

%%%%%%%%%%%%%%%%% Creamos un entorno para listar código fuente %%%%%%%%%%%%%%%
\newenvironment{sourcecode}
{\begin{list}{}{\setlength{\leftmargin}{1em}}\item\scriptsize\bfseries}
{\end{list}}

\newenvironment{littlesourcecode}
{\begin{list}{}{\setlength{\leftmargin}{1em}}\item\tiny\bfseries}
{\end{list}}

\newenvironment{summary}
{\par\noindent\begin{center}\textbf{Abstract}\end{center}\begin{itshape}\par\noindent}
{\end{itshape}}

\newenvironment{keywords}
{\begin{list}{}{\setlength{\leftmargin}{1em}}\item[\hskip\labelsep \bfseries Keywords:]}
{\end{list}}

\newenvironment{palabrasClave}
{\begin{list}{}{\setlength{\leftmargin}{1em}}\item[\hskip\labelsep \bfseries Palabras clave:]}
{\end{list}}


%%%%%%%%%%%%%%%%%%%%%%%%%%%%%%%%%%%%%%%%%%%%%%%%%%%%%%%%%%%%%%%%%%%%%%%%%%%%%%%
% Format
%%%%%%%%%%%%%%%%%%%%%%%%%%%%%%%%%%%%%%%%%%%%%%%%%%%%%%%%%%%%%%%%%%%%%%%%%%%%%%%
%\usepackage{showframe}
%\marginparwidth 0mm
%%\topmargin -4 mm
%\topmargin -21 mm
%\headheight 10 mm
%\headsep 10 mm

%\textheight 229 mm
%\textheight 246 mm

%\oddsidemargin -5.4 mm
%\evensidemargin -5.4 mm
%\oddsidemargin 5 mm
%\evensidemargin 5 mm

%\oddsidemargin -3 mm
%\evensidemargin -3 mm

%\textwidth 17 cm
%\textwidth 15 cm
%\columnsep 10 mm

\input{amssym.def}

%%%%%%%%%%%%%%%%%%%%%%%%%%%%%%%%%%%%%%%%%%%%%%%%%%%%%%%%%%%%%%%%%%%%%%%%%%%%%%%

\begin{document}

%%%%%%%%%%%%%%%%%%%%%%%%%%%%%%%%%%%%%%%%%%%%%%%%%%%%%%%%%%%%%%%%%%%%%%%%%%%%%%%
% First Page
%%%%%%%%%%%%%%%%%%%%%%%%%%%%%%%%%%%%%%%%%%%%%%%%%%%%%%%%%%%%%%%%%%%%%%%%%%%%%%%

\pagestyle{empty}
\thispagestyle{empty}


\newcommand{\HRule}{\rule{\linewidth}{1mm}}
\setlength{\parindent}{0mm}
\setlength{\parskip}{0mm}

\vspace*{\stretch{0.5}}

\begin{center}
    \includegraphics[scale=0.8]{images/etsit-logo.png}\\[10mm]
    {\Huge Trabajo Fin de Grado}
\end{center}

\HRule
\begin{flushright}
        {\Huge Framework web de computación evolutiva} \\[2.5mm]
        {\Large \textit{Web framework of evolutionary computing}} \\[5mm]
        {\Large Cristian Manuel Abrante Dorta} \\[5mm]


\end{flushright}
\HRule
\vspace*{\stretch{2}}
\begin{center}
  \Large San Cristóbal de La Laguna, \today
\end{center}

\setlength{\parindent}{5mm}

%%%%%%%%%%%%%%%%%%%%%%%%%%%%%%%%%%%%%%%%%%%%%%%%%%%%%%%%%%%%%%%%%%%%%%%%%%%%%%%
% Signature page (add the official stamp)
%%%%%%%%%%%%%%%%%%%%%%%%%%%%%%%%%%%%%%%%%%%%%%%%%%%%%%%%%%%%%%%%%%%%%%%%%%%%%%%
\newpage
%\cleardoublepage
\thispagestyle{empty}

D. {\bf Eduardo Manuel Segredo González}, con N.I.F. 78.564.242-Z,
profesor asociado adscrito al Departamento de Ingeniería Informática y de Sistemas
de la Universidad de La Laguna, como tutor,

\bigskip
D. {\bf Coromoto Antonia León Hernández}, con N.I.F. 78.605.216-W,
profesora Catedrática de Universidad
adscrita al Departamento de Ingeniería Informática y de Sistemas
de la Universidad de La Laguna, como cotutora

\bigskip
\bigskip
{\bf C E R T I F I C A N}

\bigskip
\bigskip
\bigskip
Que la presente memoria titulada:

\bigskip
``{\it Plataforma web de computación evolutiva.}''

\bigskip
\bigskip
\bigskip

\noindent ha sido realizada bajo su dirección por D. {\bf Cristian Manuel Abrante Dorta},
con N.I.F. 45.939.508-K.

\bigskip
\bigskip

Y para que así conste, en cumplimiento de la legislación vigente y a los efectos
oportunos firman la presente en La Laguna a \today

%\cleardoublepage
\newpage
%%%%%%%%%%%%%%%%%%%%%%%%%%%%%%%%%%%%%%%%%%%%%%%%%%%%%%%%%%%%%%%%%%%%%%%%%%%%%%%
\thispagestyle{empty}

{ \flushright

\begin{LARGE}
Agradecimientos
\end{LARGE}

\hspace{3mm}

\begin{large}


\hspace{3mm}
A mis tutores Eduardo y Coromoto, por aconsejarme debidamente y por la pasión que transmiten por su trabajo.

\hspace{3mm}
\hspace{3mm} \\
A mis padres y amigos, por ayudarme siempre y estar ahí cuando los necesito.


\end{large}

}

%%%%%%%%%%%%%%%%%%%%%%%%%%%%%%%%%%%%%%%%%%%%%%%%%%%%%%%%%%%%%%%%%%%%%%%%%%%%%%%%%
\newpage

\begin{huge}
Licencia
\end{huge}

\bigskip
* Si quiere permitir que se compartan las adaptaciones de tu obra mientras se comparta de la misma manera
y NO quieres permitir usos comerciales de tu obra indica:

\begin{center}
\includegraphics[scale=1.5]{images/by-nc-sa_88x31}\\[10mm]
{\Large \copyright~Esta obra está bajo una licencia de Creative Commons Reconocimiento-NoComercial-CompartirIgual 4.0 Internacional.
}
\end{center}

%%%%%%%%%%%%%%%%%%%%%%%%%%%%%%%%%%%%%%%%%%%%%%%%%%%%%%%%%%%%%%%%%%%%%%%%%%%%%%%
\newpage  %\cleardoublepage
\begin{abstract}
{\em

La computación evolutiva es un área muy prometedora en el ámbito de la Computación y la Inteligencia Artificial. Además, el mundo de las aplicaciones web está cobrando una gran relevancia hoy en día debido al surgimiento de numerosas tecnologías que aumentan la capacidad de cómputo de los navegadores, permitiendo ejecutar algoritmos y programas con un mayor coste computacional desde el propio front-end.

\bigskip

Teniendo presente estas dos ideas, en este trabajo de fin de grado se ha desarrollado un framework completo de computación evolutiva orientado a la web, utilizando el lenguaje TypeScript. La biblioteca implementada cuenta con las operaciones más comunes referentes a los algoritmos evolutivos y además es extensible para aplicaciones concretas. A su vez, se ha desarollado una aplicación web para resolver el problema de la mochila haciendo uso del framework y así ejemplificar su uso en un problema real.
}

\begin{palabrasClave}
Computación evolutiva, framework, desarrollo de software, algoritmos evolutivos.
\end{palabrasClave}

\end{abstract}
%%%%%%%%%%%%%%%%%%%%%%%%%%%%%%%%%%%%%%%%%%%%%%%%%%%%%%%%%%%%%%%%%%%%%%%%%%%%%%%

%%%%%%%%%%%%%%%%%%%%%%%%%%%%%%%%%%%%%%%%%%%%%%%%%%%%%%%%%%%%%%%%%%%%%%%%%%%%%%%
\newpage  %\cleardoublepage
\begin{summary}
{\em

Evolutionary computation is a very promising area in computer science and artificial intelligence. Furthermore, the world of web applications is becoming very significant nowadays because of the growth of many different technologies that enhance the computation power of web browsers, allowing them to execute very demanding algorithms and programs from just the front end.

\bigskip

Taking into account this two ideas, in this end of degree thesis I had developed a complete evolutionary computation framework oriented to the web, using TypeScript as programming language. The implemented library has the most common operations of evolutionary algorithms and at the same time it is extensible for specific purposes. Also, I had developed a web application for solving the knapsack problem using this framework for giving a real example for this project.
}

\begin{keywords}
Evolutionary computing, framework, software development, evolutionary algorithms
\end{keywords}

\end{summary}
%%%%%%%%%%%%%%%%%%%%%%%%%%%%%%%%%%%%%%%%%%%%%%%%%%%%%%%%%%%%%%%%%%%%%%%%%%%%%%%

%%%%%%%%%%%%%%%%%%%%%%%%%%%%%%%%%%%%%%%%%%%%%%%%%%%%%%%%%%%%%%%%%%%%%%%%%%%%%%%
\newpage{\pagestyle{empty}}
\thispagestyle{empty}

%%%%%%%%%%%%%%%%%%%%%%%%%%%%%%%%%%%%%%%%%%%%%%%%%%%%%%%%%%%%%%%%%%%%%%%%%%%%%%%


\pagestyle{myheadings} %my head defined by markboth or markright
% No funciona bien \markboth sin "twoside" en \documentclass, pero al
% ponerlo se dan un montón de errores de underfull \vbox, con lo que no se
% ha puesto.
\markboth{Cristian Manuel Abrante Dorta}{Framework web de computación evolutiva}

%%%%%%%%%%%%%%%%%%%%%%%%%%%%%%%%%%%%%%%%%%%%%%%%%%%%%%%%%%%%%%%%%%%%%%%%%%%%%%%
%Numeracion en romanos
\renewcommand{\thepage}{\roman{page}}
\setcounter{page}{1}

%%%%%%%%%%%%%%%%%%%%%%%%%%%%%%%%%%%%%%%%%%%%%%%%%%%%%%%%%%%%%%%%%%%%%%%%%%%%%%%

\tableofcontents

%%%%%%%%%%%%%%%%%%%%%%%%%%%%%%%%%%%%%%%%%%%%%%%%%%%%%%%%%%%%%%%%%%%%%%%%%%%%%%%
\newpage{\pagestyle{empty}}

\listoffigures

%%%%%%%%%%%%%%%%%%%%%%%%%%%%%%%%%%%%%%%%%%%%%%%%%%%%%%%%%%%%%%%%%%%%%%%%%%%%%%%
\newpage{\pagestyle{empty}}

\listoftables

%%%%%%%%%%%%%%%%%%%%%%%%%%%%%%%%%%%%%%%%%%%%%%%%%%%%%%%%%%%%%%%%%%%%%%%%%%%%%%%
\newpage{\pagestyle{empty}}

%%%%%%%%%%%%%%%%%%%%%%%%%%%%%%%%%%%%%%%%%%%%%%%%%%%%%%%%%%%%%%%%%%%%%%%%%%%%%%%
%Numeracion a partir del capitulo I
\renewcommand{\thepage}{\arabic{page}}
\setcounter{page}{1}


\chapter{Introducción}
\label{chapter:intro}

%%%%%%%%%%%%%%%%%%%%%%%%%%%%%%%%%%%%%%%%%%%%%%%%%%%%%%%%%%%%%%%%%%%%%%%%%%%%%
% Chapter 1: Introducción 
%%%%%%%%%%%%%%%%%%%%%%%%%%%%%%%%%%%%%%%%%%%%%%%%%%%%%%%%%%%%%%%%%%%%%%%%%%%%%%%

Este capítulo servirá para presentar una clasificación de problemas, exponiendo la complejidad que poseen algunos de ellos, lo que los hace dificilmente abordables mediante técnicas tradicionales, teniendo la necesidad de usar otras técnicas como la computación evolutiva.

%---------------------------------------------------------------------------------
\section{Clasificación de problemas}
\label{1:sec:1}

Existen una gran cantidad de problemas que se abordan hoy en día desde el área de las matemáticas, la inteligencia artificial o la ingeniería. Muchos de estos problemas pueden tener diversas aplicaciones prácticas en ámbitos muy variados, y otros muchas veces sirven como formulaciones teóricas cuyo objetivo es encontrar cuales son los límites de la tecnología. \\

\subsection{Clasificación como \textit{caja negra}}

Los sistemas computacionales encargados de resolver estos problemas, pueden entenderse como \textbf{cajas negras} (\textit{black boxes}) \cite{eiben2003introduction}. Este esquema mental que usamos para describir los sistemas encargados de resolver problemas, parte de la base de que un sistema es una caja que recibe una serie de entradas desde el exterior, y que a partir de un modelo o programa que tiene almacenado, es capaz de procesar dicho conjunto de señales de entrada para devolver una salida. \\

El nombre \textbf{caja negra} viene dado porque normalmente, este modelo que procesa las señaeles no viene especificado de manera explícita, y por tanto puede tener diversas formas: por ejemplo puede ser una ecuación o conjunto de ecuaciones que procesen una entrada numérica, o también una herramienta estadística que devuelva una estimación a partir de la entrada, o incluso puede ser un modelo lógico que ejecute una serie de sentencias para procesar señales. \\

En cualquier caso, esta \textbf{caja negra} tiene tres partes fundamentales: las entradas, el modelo de procesamiento y la salida. Además, está claro que la parte fundamental es el modelo de computación, que de ser conocido nos permitiría calcular la salida para cualquier entrada al sistema. \\

\begin{figure}[ht]
    \centering
    \includegraphics[scale=0.6]{mem/images/cap-1/1-1.png}
    \caption{Esquema general de un modelo computacional de caja negra}
    \label{fig:my_label}
\end{figure}

Este esquema que hemos definido es muy conveniente para establecer un criterio de clasificación de problemas, en función de que partes del sistema son conocidas y cuales no. A partir de esto podemos diferenciar en tres tipos de problemas:

\subsubsection{Optimización}
Los problemas de optimización son aquellos en los que se conoce el modelo, además de la salida que se espera, o al menos una descripción de la misma, y en función de ello, se debe calcular cuales son los valores de entrada que proporcionan dicha salida. \\

Existen multitud de problemas que se clasifican en esta categoría, aunque quizás uno de los más clásicos es el problema de viajante de comercio (\textbf{TSP}). Este problema consiste en encontrar la secuencia de rutas de coste mínimo dado un grafo de ciudades y sus interconexiones, de tal forma que cada una de las ciudades sea visitada solo una vez.\\

Como vemos, en este problema está especificada cual es la salida esperada, además del modelo de computación, sin embargo, lo que desconocemos es cual será la entrada, es decir, la combinación de rutas que minimizará el coste y que satisface las restricciones.

\subsubsection{Modelización}

En los problemas de modelización, se conocen las entradas y sus correspondientes salidas, pero se desconoce cual es el modelo de computación que debe usarse para procesarlas. \\


Este es el tipo de problemas que se abordan en áreas como el \textbf{Machine Learning}, pues en ellos se suele tener un conjunto de datos en los que existe una correspondecia entre las entradas y su resultado esperado. El problema está en crear un modelo que sea capaz de \textit{aprender} a partir de dichos datos y por tanto, sea capaz de generalizar y extraer características de datos que no haya procesado previamente.

\subsubsection{Simulación}

Este es el tipo de problemas más lineales, pues en ellos se conoce cuales serán las entradas y el modelo de computación, pero se desconoce cual será la salida. \\

Existen multitud de ejemplos de problemas de simulación, como por ejemplo; simulación de fluidos, o simulación meteorológica. Este tipo de problemas tienen una gran utilidad, pues nos permiten predecir una realidad futura, lo cual es crucial en múltiples ámbitos. \\

%------

\subsection{Clasificación en función de la complejidad (Clases P y NP)}

Otra clasificación posible que se puede hacer de los problemas es en función de la \textit{complejidad} que entraña resolverlos. De esta forma una clasificación básica en este sentido pueden ser los problemas \textit{fáciles} y \textit{difíciles} de resolver \cite{garey2002computers}.\\

Esta definición puede resultar ambigua, puesto que existen múltiples algoritmos que pueden resolver un problema. Es por ello, que para medir la complejidad de un problema, elegimos el algoritmo que lo resuelve en menor tiempo. Con \textbf{tiempo de ejecución} realmente no nos referimos a una magnitud física, sino al número de pasos elementales u operaciones que lleva finalizarlo. \\

Atendiendo a esta clasificación, nos encontramos 3 clases de problemas \cite{papadimitriou2003computational}:

\begin{itemize}
    \item \textbf{Clase $\mathcal{P}$}: Los problemas que pertenecen a la clase $\mathcal{P}$, son aquellos en los que existe un algoritmo que pueda resolverlos en tiempo polinomial. Es decir que la función que expresa la complejidad temporal de resolución del algoritmo, es un polinomio.
    \item \textbf{Clase $\mathcal{NP}$}: Para los problemas que están en $\mathcal{NP}$ no existe un algoritmo que pueda resolverlos en tiempo polinomial, sin embargo, se puede evaluar si una solución es válida para el problema en tiempo polinomial. Por ejemplo, el TSP o el VRP son problemas que pertenecen a esta clase.
    \item \textbf{Clase $\mathcal{NP}-Completo$}: Este es un subconjunto de problemas de $\mathcal{NP}$, que se corresponden con los problemas mas difíciles de resolver. Se considera que un problema es $\mathcal{NP}$-Completo cuando está en $\mathcal{NP}$ y además, \textbf{todos} los problemas de $\mathcal{NP}$ se pueden transformar a dicho problema mediante una \textbf{reducción polinomial}. Encontrar un problema que pertenenciera a esta clase fue una tarea compleja hasta el desarrollo del \textbf{Teorema de Cook} \cite{garey2002computers}.
\end{itemize}

Uno de los problemas que más trae de cabeza a la comunidad científica, considerado como una de las grandes incógnitas de la matemática moderna es el problema ¿$\mathcal{P} = \mathcal{NP}$? Este problema trata de discernir si verdaderamente existen problemas que están en $\mathcal{NP}$ \cite{baker1975relativizations} porque tienen una naturaleza diferente a los que están en $\mathcal{P}$, o es porque aún no hemos encontrado un algoritmo lo sufiecientemente bueno para resolverlo en un tiempo razonable.

%---------------------------------------------------------------------------------
\section{Computación evolutiva}
\label{1:sec:2}

Como podemos ver, existe una gran cantidad de problemas que no poseen un algoritmo exacto que los pueda resolver en un tiempo razonable. Es por ello, que se necesita buscar alternativas que nos permitan llevar a cabo la resolución de dichos problemas, aunque muchas veces se deba hacer de manera aproximada. \\

La computación evolutiva es una técnica que nos permite encontrar las soluciones a dichos problemas complejos, inspirándose en el proceso evolutivo natural para lograr los resultados. El proceso natural de evolución tiene mucho potencial a la hora de construir buenas soluciones a los problemas, principalmente porque se potencia el mecanismo de "\textit{prueba y error}" \cite{holland1973genetic}. \\

En el proceso evolutivo natural, contamos con un ambiente en el que se encuentran una serie de recursos, por los cuales tienen que competir los individuos que se encuentren en él. Además de tratar de obtener la mayor parte de los recursos disponibles, los individuos tratarán de reproducirse entre sí. De esta forma, en las sucesivas generaciones de individuos, solo los más daptados habrán conseguido sobrevivir y transmitir sus características físicas a su descendencia, a través del material genético. \\

Mediante un algoritmo evolutivo, se trata de simular este mismo procedimiento. Para ello, consideramos que el ambiente que contiene los recursos por los que los individuos deben competir es el problema que queremos solucionar, y las posibles soluciones del problema son los individuos. Para simular el grado de adaptación al ambiente que deben tener estos individuos se utilizará una función que medirá cual buena es la solución que se está evaluando. Al igual que ocurre en un ambiente real, estos individuos deberán reproducirse entre sí, transmitiendo su material \textit{genético} a la descendencia. \\

%---------------------------------------------------------------------------------
\section{Algoritmos evolutivos}
\label{1:sec:3}

Siguiendo la idea básica que se quiere perseguir con los algoritmos evolutivos, podemos construir un esquema para determinar cuales serán las distintas fases que tendrán este tipo de algoritmos:

\begin{algorithm}[H]
 INICIALIZAR la población con $n$ individuos aleatorios;\\
 EVALUACIÓN de la población mediante la función de fitness;
 
 \While{CONDICIÓN DE PARADA no sea satisfecha}{
  SELECCIÓN de padres; \\
  RECOMBINACIÓN de pares de padres; \\
  MUTACIÓN de la descendencia; \\
  EVALUCACIÓN de la descendencia; \\
  SELECCIÓN de supervivientes para la siguiente generacón;
 }
 \caption{Esquema básico de un algoritmo evolutivo}
\end{algorithm}

En este esquema podemos ver las diferentes fases de las que se compone un algoritmo evolutivo. En primer lugar, se deberá inicializar una población con individuos generados aleatoriamente, que se corresponderán con las posibles soluciones que puede tener nuestro problema, a continuación estás soluciones deberán ser evaluadas mediante la función de fitness, determinando así su grado de adaptación al medio. Seguidamente, se entrará en el bucle principal del algoritmo, el cual continuará ejecutándose mientras que una determinada condición de parada no haya sido satisfecha. Dentro de este bucle, se aplicarán los operadores principales sobre la población generada; en primer lugar, se seleccionarán los padres que se reproducirán en la siguiente generación, a los cuales se les aplicará un operador de cruce para generar una descendencia. Esta descendencia, sufrirá de manera aleatoria una mutación que afectará a su material genético, y después de ello, se evaluará de nuevo respecto a la función de fitness. Seguidamente, se deberán seleccionar que individuos son los que permanecerán en la siguiente generación, repitiendo de nuevo el bucle. \\ 

Podemos afirmar que este es un procedimiento estocástico, puesto que en él están involucrados numerosos componentes aleatorios. Esta aleatoriedad es el motor principal de las dos fuerzas que garantizan la eficacia de este tipo de algoritmos \cite{eiben2003introduction}:

\begin{itemize}
    \item \textbf{Variación}: Los operadores de variación (recombinación y mutación), generan la suficiente diversidad como para explorar una gran parte del espacio de soluciones.
    \item \textbf{Intensificación}: Los operadores de selección tanto de padres como de descendencia, son los responsables de que se exploren las mejores soluciones garantizando que su material genético perviva en sucesivas generaciones.
\end{itemize}

Para cada una de las fases de las que se compone un algoritmo evolutivo, existen numerosos métodos que se pueden aplicar. Los cuales dependen enormemente de la codificación que se le esté aplicando a los individuos, además del problema concreto que se esté tratando de resolver.

%%%%%%%%%%%%%%%%%%%%%%%%%%%%%%%%%%%%%%%%%%%%%%%%%%%%%%%%%%%%%%%%%%%%%%%%%%%%%%%

\chapter{Desarrollo y tecnologías utilizadas}
\label{chapter:desarrollo}

%%%%%%%%%%%%%%%%%%%%%%%%%%%%%%%%%%%%%%%%%%%%%%%%%%%%%%%%%%%%%%%%%%%%%%%%%%%%%%%
% Chapter 2: Antecedentes
%%%%%%%%%%%%%%%%%%%%%%%%%%%%%%%%%%%%%%%%%%%%%%%%%%%%%%%%%%%%%%%%%%%%%%%%%%%%%%%

%++++++++++++++++++++++++++++++++++++++++++++++++++++++++++++++++++++++++++++++

\section{}
\label{:sec1}



%%%%%%%%%%%%%%%%%%%%%%%%%%%%%%%%%%%%%%%%%%%%%%%%%%%%%%%%%%%%%%%%%%%%%%%%%%%%%%%
\newpage{\pagestyle{empty}}
\thispagestyle{empty}

\chapter{Modo de uso de genetics.js}
\label{chapter:casodeuso}

%%%%%%%%%%%%%%%%%%%%%%%%%%%%%%%%%%%%%%%%%%%%%%%%%%%%%%%%%%%%%%%%%%%%%%%%%%%%%%%
% Chapter 3: Objetivos
%%%%%%%%%%%%%%%%%%%%%%%%%%%%%%%%%%%%%%%%%%%%%%%%%%%%%%%%%%%%%%%%%%%%%%%%%%%%%%%

%++++++++++++++++++++++++++++++++++++++++++++++++++++++++++++++++++++++++++++++


La carencia de \textit{frameworks} de computación evolutiva adaptados a la web es la motivación principal que se encuentra detrás de este trabajo. Pero para lograr que se lleve a cabo el desarrollo de este proyecto, se deben establecer una serie de objetivos y tareas a realizar. \\

En primer lugar decidiremos un nombre para el proyecto y el \textit{framework}: \textbf{genetics.js}. La elección de este nombre está inspirada en uno de los tipos principales de algoritmos evolutivos; los algoritmos genéticos, además de añadirle el sufijo \textit{.js}, en referencia al lenguaje \textbf{JavaScript}, en el cual se llevará a cabo el desarrollo. \\

De esta forma, estableceremos las características que deseamos que poseea \textbf{genetics.js}, cuya implementación serán los objetivos principales del proyecto:

\begin{itemize}
    \item \textbf{Estar desarrollado en un lenguaje moderno que tenga una gran comunidad}: Es importante que el lenguaje en el que se desarrolle el proyecto nos permita ser ejecutado en un entorno web y a ser posible en otros entornos. Por otra parte que tenga una gran comunidad de usuarios, para así poder llevar el proyecto a una mayor cantidad de gente.
    \item \textbf{Poseer una buena documentación}: La documentación es una tarea fundamental, puesto que las diferentes técnicas y métodos disponibles deben estar bien explicados de tal forma que puedan ser comprensibles por los usuarios.
    \item \textbf{Establecer herramientas y procedimientos para garantizar la estabilidad y continuidad del proyecto}: Para garantizar la estabilidad y continuidad del proyecto se pueden utilizar herramientas como los tests, la integración continua o el control de versiones.
    \item \textbf{Implementación de la mayoría de métodos comunes}: Se debe poseer una implementación de los métodos más comunes en las diversas fases de un algoritmo evolutivo.
    \item \textbf{Capacidad de extensión}: Para que el framework pueda ser extensible para problemas concretos, se debe utilizar mecanismos de software como la herencia, o la implementación de interfaces.
\end{itemize}

En función de estos criterios, se ha establecido un mapa de desarrollo que se corresponderá con las versiones que se publicarán de la librería, atendiendo al versionado semántico (\textbf{semantic versioning} \cite{semver}):

\begin{itemize}
    \item \texttt{0.1.0}: Implementación de la codificación de soluciones mediante \textbf{individuos}.
    \item \texttt{0.2.0}: Implementación de los operadores de mutación.
    \item \texttt{0.3.0}: Impelementación de los operadores de cruce.
    \item \texttt{0.4.0}: Implementación de los operadores de selección de padres. 
    \item \texttt{0.5.0}: Implementación de los operadores de selección de supervivientes.
    \item \texttt{0.6.0}: Implementación de las clases gestoras de la población de individuos.
\end{itemize}

%%%%%%%%%%%%%%%%%%%%%%%%%%%%%%%%%%%%%%%%%%%%%%%%%%%%%%%%%%%%%%%%%%%%%%%%%%%%%%%
\newpage{\pagestyle{empty}}
\thispagestyle{empty}

\chapter{Conclusiones y líneas futuras}
\label{chapter:conclusiones}

%%%%%%%%%%%%%%%%%%%%%%%%%%%%%%%%%%%%%%%%%%%%%%%%%%%%%%%%%%%%%%%%%%%%%%%%%%%%%%%
% Chapter 4 : Desarrollo y tecnologías utilizadas
%%%%%%%%%%%%%%%%%%%%%%%%%%%%%%%%%%%%%%%%%%%%%%%%%%%%%%%%%%%%%%%%%%%%%%%%%%%%%%%



%++++++++++++++++++++++++++++++++++++++++++++++++++++++++++++++++++++++++++++++

En este capítulo se describirá en profundidad el framework \textbf{genetics.js}. Se expondrán tanto las tecnologías utilizadas, justificando debidamente la elección, así como la propia estructura que tiene el software que se ha desarrollado.

%---------------------------------------------------------------------------------
\section{Tecnologías utilizadas}
\label{4:sec:1}

Dado que el objetivo fundamental de este proyecto es el desarrollo de un framework de computación evolutiva que sea completamente compatible con la web, las tecnologías más apropiadas para este desarrollo serán las que se utilicen en el \textit{stack} del lenguaje JavaScript. \\

Por ello, en primer lugar llevaremos a cabo una introducción a dicho lenguaje de programación, para luego exponer las dependecias externas que se ha utilizado durante la fase de desarrollo y las que se han elegido para ser utilizadas en la versión de la librería en producción, es decir las que serán utilizadas por los usuarios finales. 

\subsection{\textit{Stack} de desarrollo en JavaScript}

En primer lugar, es importante introducir el lenguaje de programación JavaScript y la importancia que tiene hoy en día, exponiendo las tecnologías más comunes que tiene aparejadas el desarrollo de una aplicación con este lenguaje. \\

JavaScript es un lenguaje de programación interpretado, multiparadigma y de tipado débil, desarrollado por Brendan Eich, durante su trabajo en Netscape, para ser utilizado por el navegador que la empresa estaba creando. Durante esa época y en sus primeros años, se consideraba un lenguaje menor, es decir que solo se utilizaba para implementar ciertos aspectos de la interacción del usuario con la página web, o para llevar a cabo operaciones sencillas en el lado del cliente. \\

Debido a la poca importancia que se le dio a su desarrollo desde el momento inicial, son destacables los grandes errores de diseño con los cuales cuenta \cite{KennethEng2019}, y los que hacen que sea bastante complejo confiar en que el software desarrollado en esta tecnología cumplirá ciertos criterios de calidad. Es por ellos que diversas empresas e instituciones, han tratado de estandarizar y complementar el lenguaje para garantizar su estabilidad y escalabilidad. Ejemplo de ello, es la organización ECMA con los estándares de JavaScript \cite{ecmascript}, o Microsoft con la creación del lenguaje TypeScript.\\

Con los años ha ganado bastante popularidad, gracias en parte a proyectos como NodeJS \cite{node}, el cual trata de convertir a JavaScript en un lenguaje con mucho más ámbito que el que tenía anteriormente, dando la posibilidad de construir un servidor completo con este lenguaje. NodeJS es una de las tecnologías más punteras para el desarrollo de servidores hoy en día, debido a su gran escalabilidad y a que soporta una gran cantidad de conexiones simultáneas, en parte grecias al uso del motor V8 de JavaScript \cite{motor-v8}, desarrollado por Google. \\

En este sentido, es muy destacable también NPM (Node Package Manager) \cite{npm} como gestor de dependencias de NodeJS. Este gestor de paquetes es el ejemplo perfecto de sencillez y eficacia, al permitir publicar nuestros propios módulos en un portal que los aglutina de manera centralizada, y que nos permite instalar, gestionar y utilizar dichos módulos de manera sencilla en nuestra propia aplicación desde la línea de comandos. \\

La gestión de dependencias es una tarea compleja que puede acarrear ciertos problemas, sobretodo de retrocompatibilidad entre versiones. Una de las grandes ventajas de NPM es que esta tarea es bastante sencilla, centralizando todas las dependecias en un fichero \textit{json} (\textbf{package.json}), en el cual se especifica el nombre del paquete y la versión que tenemos instalada. De esta forma se garantiza que se está utilizando en nuestra aplicación exactamente la dependencia que queremos. \\

Además, el versionado de los paquetes se basa en \textbf{semantic versioning} \cite{semver}, contando con la posibilidad de distinguir entre versiones \textbf{minor, major y patch}, garantizando así que se pueda seguir el mapa de desarrollo previsto. \\

De esta forma, teniendo el potencial de una herramienta como NPM, la posibilidad de llevar esta idea a aplicaciones cliente es bastante interesante, puesto que para la web la importación de módulos externos no se gestiona de una manera tan eficaz que como se hace con NodeJS y NPM. Es ahí donde entran herramientas como Webpack y Babel en juego, las cuales permiten que el código que se importa mediante NPM y se utiliza en ficheros de código fuente, sea compilado para ser utilizado directamente en el \textit{front-end}. De esta forma podemos utilizar NPM como gestor de dependencias aunque estemos trabajando en el lado del cliente. \\

Como vemos, la existencia de este tipo de tecnologías hace que sea muy conveniente desarrollar la librería \textbf{genetics.js} como un módulo NPM, puesto que ya no solo podría ser utillizada en el lado del cliente, sino que también hace posible que se utilice en otros ámbitos como un servidor con NodeJS o cualquier otra tecnología basada en JavaScript.

\subsection{Tecnologías utilizadas para el desarrollo}

Tal y como se ha comentado, la librería \textbf{genetics.js} se desarrollará como un módulo NPM para garantizar que sea compatible con tecnologías web. En este primer apartado, expondremos cuales seran las dependencias que este módulo tendrá en el desarrollo. Estas dependencias realmente no afectarán al usuario final, peusto que no serán descargadas ni utilizadas cuando se instale el paquete, ya que solo son útiles para garantizar y facilitar el desarrollo correcto de la librería. \\

Las tecnologías que se han utilizado como dependencias de desarrollo han sido

\subsection{Tecnologías utilizadas en producción}

%%%%%%%%%%%%%%%%%%%%%%%%%%%%%%%%%%%%%%%%%%%%%%%%%%%%%%%%%%%%%%%%%%%%%%%%%%%%%%%
\newpage{\pagestyle{empty}}
\thispagestyle{empty}

\chapter{Summary and Future lines}
\label{chapter:ingles}

%%%%%%%%%%%%%%%%%%%%%%%%%%%%%%%%%%%%%%%%%%%%%%%%%%%%%%%%%%%%%%%%%%%%%%%%%%%%%
% Chapter 5: Caso de uso 
%%%%%%%%%%%%%%%%%%%%%%%%%%%%%%%%%%%%%%%%%%%%%%%%%%%%%%%%%%%%%%%%%%%%%%%%%%%%%%%

%++++++++++++++++++++++++++++++++++++++++++++++++++++++++++++++++++++++++++++++

En este capítulo se expondrá una aplicación concreta implementada con \textbf{genetics.js}, la librería desarrollada. En esta implementación se ha construido una aplicación web usando React como framework de creación de interfaces de usuario, además de usar la librería para ejecutar un algoritmo evolutivo en el lado del cliente. \\

La intención que tiene este desarrollo es mostrar la capacidad de la que dispone la librería para ser ejecutada en el \textit{front-end}, junto a tecnologías modernas para la creación de páginas web. \\

\section{Descripción del problema a implementar}

La aplicación que se ha implementado es un simulador del problema de la mochila (\textit{Knapsack problem}), resuelto mediante un algoritmo evolutivo. La intención principal es que se puedan seleccionar la mayoría de parámetros posibles del algoritmo, para así poder ver como cambia el rendimiento con diferentes configuraciones de entrada. \\

Pero antes de describir el procedimiento concreto para desarrollar la aplicación, es importante realizar una definición del problema que se pretende abordar. Cabe destacar que el problema de la mochila es clásico en el ámbito de la optimización combinatoria \cite{karp1972reducibility}, y ha sido objeto de estudio desde el nacimiento del campo. \\

En la formulación del problema disponemos de $n$ items que pueden ser introducidos en la mochila. Cada item tiene dos atributos, su peso ($w_i$) y su valor o beneficio ($v_i$). De esta forma, la solución al problema trata de descubrir cual es el conjunto de items que debe introducirse en la mochila de tal forma que se maximize el beneficio total pero sin sobrepasar la capacidad ($W$) de la mochila. \\

De manera formal se puede describir el problema de esta forma:

\begin{equation*}
\begin{aligned}
& \underset{X}{\text{max}}
& & \sum_{i=1}^n v_i x_i \\
& \text{sujeto a}
& & \sum_{i=1}^n w_i x_i \leq W \\
&&& x_i \in \{0, 1\}
\end{aligned}
\end{equation*}

Donde $X = \{x_1, \dots, x_n\}$ es el conjunto de variables de decisión que establecen si el item $i$-esimo está o no presente en la mochila. \\

Existen múltiples técnicas para abordar este problema, incluso pudiendo ser resuelto mediante una técnicas exactas. Pero con motivo de probar la librería desarrollada, el problema se resolverá con un algoritmo genético. \\

La codificación que se utilizará para los individuos es la \textbf{codificación binaria}, donde cada uno de los genes del individuo representará si el ítem en la posición establecida se encuentra o no en la mochila. \\

Los parámetros del algoritmo que podrán seleccionarse serán:

\begin{itemize}
    \item \textbf{Tamaño de la población}: El tamaño de la población deberá estar comprendido entre 0 y 25. Donde el límite superior se ha establecido por criterios de claridad en la visualización.
    \item \textbf{Tipo de operador de cruce}: El operador de cruce que se puede seleccionar puede ser el crossover de un punto (\texttt{OnePointCrossover}), el crossover de n puntos (\texttt{NPointsCrossover}) o el crossover uniforme (\texttt{UniformCrossover}). Cabe destacar que dependiendo del operador elegido, se deberán establecer sus parámetros.
    \item \textbf{Método de selección de padres}: Se puede elegir entre \textit{roulette wheel} y \textit{stochastic universal sampling}.
    \item \textbf{Método de reemplazo}: Se puede seleccionar entre el remplazo basado en la edad (\texttt{AgeBasedReplacement}) y el basado en el fitness (\texttt{FitnessBasedReplacement}).
    \item \textbf{La tasa de mutación}: La tasa de mutación (\textit{mutation rate}), puede establecerse, teniendo 0.5 como valor por defecto.
    \item \textbf{Numero máximo de generaciones}: El número máximo de generaciones será también un parámetro a establecer, aunque por defecto son 50.
\end{itemize}

\section{Desarrollo de la aplicación}

Para implementar la aplicación web, hemos utilizado \textbf{React} \cite{react} como librería para la creación de la interfaz de usuario, y \textbf{TypeScript} como lenguaje de desarrollo. Esta elección se ha tomado porque es mucho más cómodo y flexible trabajar con una librería como React para crear la interfaz de usuario en lugar de trabajar con HTML y CSS directamente. \\

Además, la metodología de trabajo de React nos permite desarrollar a partir de componentes, lo que añade una gran modularidad al diseño, simplificando la tarea de mostrar la información referente al algoritmo que pretendemos ejecutar. \\

Para simplificar la tarea de gestionar diferentes elementos visuales se ha utilizado la librería \textbf{Material UI} \cite{materialui}, que provee una serie de componentes inspirados en el estandard Material Design, desarrollado por Google. Como la aplicación cuenta con bastantes formularios diferentes, se ha utilizado la librería \textbf{Formik} \cite{formik} para la gestión, además de usar \textbf{Yup} \cite{yup} para garantizar la validez de los datos introducidos. \\

En cuanto al diseño visual de la aplicación, cabe destacar que contamos con dos pantallas, la primera utilizada para establecer los parámetros del algoritmo y la segunda para mostrar la ejecución del mismo. Explicaremos ambas pantallas por separado. \\

En primer lugar, expondremos la pantalla principal. En ella podremos seleccionar todos los parámetros referentes al algoritmo además de establecer los items que se encontrarán en la mochila, así como su capacidad. \\

\begin{figure}[H]
    \centering
    \includegraphics[scale=0.2]{mem/images/cap-5/knapsack-1.png}
    \caption{Pantalla principal de la aplicación web desarrollada}
    \label{fig:screenshot-1}
\end{figure}

En la parte central de la pantalla tendremos el formulario en el que se muestran los items que formarán parte de la mochila. Podremos introducir items y también eliminarlos, pudiendo a su vez establecer el peso y cantidad del item en cuestión. Además, en la parte derecha contamos con un campo de texto en el que podremos introducir la capacidad de la mochila. Cabe señalar que ambos formularios están verificados para impedir que se introduzcan valores erróneos. \\

El segundo formulario con el que nos encontramos es el de parámteros del algoritmo. En el podremos elegir cuales serán las operaciones que se aplicarán a la hora de ejecutar el algoritmo genético. Al igual que ocurría en el caso anterior, en este formulario también se ha llevado a cabo una verificación para evitar los valores erróneos. \\

Una vez pulsamos el boton \textbf{execute algorithm} de la parte inferior, la aplicación nos llevaría a una segunda pantalla en la que se mostraría la ejecución del algoritmo.

\begin{figure}[H]
    \centering
    \includegraphics[scale=0.2]{mem/images/cap-5/knapsack-2.png}
    \caption{Pantalla de ejecución del algoritmo}
    \label{fig:screenshot-1}
\end{figure}

Esta pantalla está dividida en una serie de secciones diferentes. En la parte izquierda, contamos con una tabla en la que se muestran las estadísticas del algoritmo evolutivo, actualizadas durante la ejecución. Entre las estadísticas que se muestran encontramos: el número de generaciones, la edad media de los individuos, o el fitness medio de la población. \\

A la izquierda de este recuadro se puede ver una representación visual de la población, la cual se irá actualizando a medida que el algoritmo se ejecute. La visualización de este apartado es la que impide que se tengan tamaños de población muy grandes. \\

Finalmente y en la parte inferior, nos encontramos una representación del individuo con un mayor valor de fitness, indicando cuales son los items que ha escogido para introducir en la mochila, así como el peso y valor que tendrá dicha selección. \\

La ejecución del algoritmo se puede hacer de manera contínua, ejecutando de una sola vez el algoritmo hasta que se llegue a un máximo de generaciones, o también se puede hacer paso a paso, por si se quiere depurar el resultado. \\

El despliegue de esta aplicación se ha hecho en \textbf{GitHub pages}, y por tanto está disponible en este enlace: \url{https://cristianabrante.github.io/GeneticsJsKnapsack/}.


%%%%%%%%%%%%%%%%%%%%%%%%%%%%%%%%%%%%%%%%%%%%%%%%%%%%%%%%%%%%%%%%%%%%%%%%%%%%%%%
\newpage{\pagestyle{empty}}
\thispagestyle{empty}

\chapter{Presupuesto}
\label{chapter:presupuesto}

%%%%%%%%%%%%%%%%%%%%%%%%%%%%%%%%%%%%%%%%%%%%%%%%%%%%%%%%%%%%%%%%%%%%%%%%%%%%%
% Chapter 5: Conclusiones y Trabajos Futuros 
%%%%%%%%%%%%%%%%%%%%%%%%%%%%%%%%%%%%%%%%%%%%%%%%%%%%%%%%%%%%%%%%%%%%%%%%%%%%%%%

%++++++++++++++++++++++++++++++++++++++++++++++++++++++++++++++++++++++++++++++

Tras haber completado el desarrollo de este proyecto, se debe hacer un balance para determinar cual ha sido el grado de satisfacción con el resultado, y de esta forma señalar si la se ha logrado la consecución de los objetivos y establecer cuales serán las líneas futuras de desarrollo.  \\

En primer lugar, cabe destacar la dificultad que entraña el diseño e implementación de un framework de computación evolutiva. Esta dificultad viene determinada sobretodo por la importancia actual que tiene este campo dentro de la computación y la inteligencia artificial, por lo cual existen una gran cantidad de aplicaciones diferentes que desarrollan nuevos métodos y técnicas cada vez más precisas y avanzadas. De esta forma, crear una librería que sea lo suficientemente flexible como para adaptarse correctamente a estos cambios, pero que a la vez disponga de una implementación concreta para las funciones más comunes entraña dificultades muy importantes. \\

De esta forma, considero que los métodos y técnicas implementados cubren la mayoría de aplicaciones básicas, aunque para abarcar muchas más técnicas y métodos, se debería intentar utilizar esta librería para problemas más avanzados. De esta forma se determinarían las carencias existentes y podrían ser solucionadas de manera paulatina. \\

En cuanto a las tecnologías de desarrollo, creo que la mayoría de decisiones han sido acertadas. Por una parte, porque se han elegido herramientas actuales, que cuentan con un gran número de usuarios, lo cual garantiza en mayor medida su estabilidad. Y por otra parte, porque al realizar la implementación de la libería adaptada a la web se puede garantizar que se pueda utilizar esta librería con las aplicaciones que se desarrollen en un futuro. Además, el desarrollo como librería web, es el más flexible pues ya no solo podrá ejecutarse en el \textit{front-end}, sino que también nos permite ser ejecutada en un servidor. \\

A continuación, pasaré a describir las lineas futuras que tiene este trabajo, las cuales he decidido dividir en varias categorías, para así distinguir que trabajo se llevará a cabo en cada punto concreto. \\

En primer lugar expondrá las lineas futuras en cuanto a \textbf{tecnologías utilizadas}:

\begin{itemize}
    \item \textbf{Utilizar un monorepo para estructurar el repositorio}: La necesidad de utilizar un monorepo es gestionar mejor las diferentes partes de las que se compone este proyecto: por una parte tendríamos la propia librería, luego la web con la documentación y finalmente un paquete con ejemplos de implementaciones concretas para problemas comunes como el TSP o el problema de la mochila. Para manejar este \textit{monorepo} se puede utilizar la tecnología \textbf{Lerna} y \textbf{Yarn} en lugar de NPM para manejar la instalación de dependencias y la ejecución de scripts.
    \item \textbf{Cambio de tecnología de documentación}: La elección de \textbf{TypeDoc} como instrumento de documentación no creo que haya sido una mala idea. Sin embargo, creo que este estilo de documentación no puede ser escalable al futuro, sino que se necesita hacer una página web a medida para la documentación, similar a las que existen en otros proyectos de grandes dimensiones. La tecnología a la que propongo realizar la migración es \textbf{Docusaurus}, principalmente por su gran integración con \textbf{React} y \textbf{Gatsby}.
\end{itemize}

En segundo lugar, estas serán las líneas futuras que se seguirán en cuanto al \textbf{desarrollo}.

\begin{itemize}
    \item \textbf{Reestructurar la clases desarrolladas más recientemente}: Por motivo de escasez de tiempo, los métodos implementados en último lugar (selección y criterio de finalización), no se han hecho con el mismo trabajo de diseño y desarrollo que las otras, con lo cual necesitarían una reestructuración.
    \item \textbf{Reestructurar la organización de ficheros}: En muchos casos existe una mala organización de ficheros, lo cual debe corregirse.
    \item \textbf{Mejora de los tests}: Actualmente, para testear la librería se utiliza una batería de tests, pero aun no existen suficientes ejemplos como para garantizar el correcto funcionamiento en todos los módulos.
    \item \textbf{Implementación de un CLI}: Creo que una buena idea es que se permitiera ejecutar algunos algoritmos mediante la linea de comandos, pudiendo elegir los parámetros de manera sencilla y garantizando que existe una correcta visualización en la terminal. Para este propósito, propongo la herramienta \textbf{enquirer}
    \item \textbf{Soporte de asincronía}: Los algoritmos evolutivos son una tarea que requiere de una gran carga computacional, así que sería un gran avance que estas tareas se pudieran ejecutar de manera asíncrona.
\end{itemize}

Finalmente, ofreceré unas líneas futuras en cuanto a la \textbf{algoritmia y nuevas funcionalidades}:

\begin{itemize}
    \item \textbf{Permitir tests estadísticos por parte del usuario}: Una tarea fundamental para evaluar un algoritmo evolutivo es ejecutar tests estadísticos sobre el resultado, es una tarea que se le debe facilitar al usuario.
    \item \textbf{Implementar individuos de permutación}: Este tipo de individuos son ampliamente utilizados, creo que es importante ofrecer una implementación a medida para ellos.
    \item \textbf{Considerar individuos numéricos multirango}: Por ahora, los individuos numéricos solo pueden tener un rango, sería interesante poder extender ese comportamiento.
    \item \textbf{Considerar individuos no factibles y permitir operadores de reparación}: Actualmente no se consideran los individuos no factibles. Creo que se deberían añadir en un futuro además de diseñar operadores específicos que reparen soluciones no factibles.
    \item \textbf{Implementar otras categorías de algoritmos}: Implementar otras categorías de algoritmos evolutivos como los algoritmos meméticos sería muy útil para los usuarios.
\end{itemize}


%%%%%%%%%%%%%%%%%%%%%%%%%%%%%%%%%%%%%%%%%%%%%%%%%%%%%%%%%%%%%%%%%%%%%%%%%%%%%%%

%%%%%%%%%%%%%%%%%%%%%%%%%%%%%%%%%%%%%%%%%%%%%%%%%%%%%%%%%%%%%%%%%%%%%%%%%%%%%%%
\addcontentsline{toc}{chapter}{Bibliografía}
\bibliographystyle{plain}

\bibliography{memtfg}
\nocite{*}

%%%%%%%%%%%%%%%%%%%%%%%%%%%%%%%%%%%%%%%%%%%%%%%%%%%%%%%%%%%%%%%%%%%%%%%%%%%%%%%

\end{document}
